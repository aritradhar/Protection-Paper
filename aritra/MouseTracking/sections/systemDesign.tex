\section{\name}
\label{sec:systemDesign}


In this section, we provide the technical details of \name integrity protection for IO devices. \name is built upon some base components such as ensuring pointer integrity, the integrity of the user interaction and UI elements, creating the communication channel between the \device and the remote server, etc. The most crucial of all is to how to preserve user's input on UI elements and preserve the integrity of the UI itself.

\subsection{Set up}
\label{sec:systemDesign:setup} 

\iffalse
\subsection{System Components}
\label{sec:systemDesign:components}

Figure~\ref{fig:approachOverview} provides the high-level approach overview of \name. The components of \name are the following:

\begin{mylist}
  \item \textbf{Host system.} The host system is completely compromised (hardware, OS and the installed applications) by the attacker.
  \item \textbf{\device.} The \device is connected to the input devices and sits between the host and the display. The \device is connected to the input devices over the \usb interface and connected to the host and the display over HDMI. Note that the \device lacks any networking capability to minimize the TCB. Additionally, the \device can emulate itself as a composite \usb device to the host.
  \item \textbf{Input device.} The input devices such as the mouse, keyboard, touchpad, etc. are connected to the \device over a \usb interface. We assume that the input devices are trusted.
  \item \textbf{Display.} The display device is connected to the \device over HDMI interface.
  \item \textbf{Remote server.} The remote server is a trusted entity that serves the web application to the host system over \http. The remote server also creates a secure channel with the \device using the host as an untrusted transport. 
  
\end{mylist}

For all these components to work in tandem, the \device and the remote server requires a communication channel. As described before, the \device does not have any networking capability. Only using the HDMI and the keyboard emulation, the \device creates a bidirectional channel with the remote server. The details of the communication channel is discussed in Section~\ref{sec:systemDesign:communicationChannel}.
\fi


%\subsection{Output Integrity}

\iffalse
\subsection{Communication Channel}
\label{sec:systemDesign:communicationChannel}

The communication channel between the \device and the remote server is constructed without employing any and separate hardware/software or web-browser extensions. \name two-way communication works out-of-the-box using JavaScript code and QR code decoding routine that runs on the \device. The \emph{downstream} channel, i.e., the data channel from the remote server and the \device relies on the HDMI channel. The remote server encodes the information in the webpage in QR code that is intercepted by the \device from the HDMI channel and decoded.

The \emph{upstream} channel, i.e., the data from the \device to the remote server is transmitted using the \name JavaScript snippet that is served by the remote web server. The \name JavaScript snippet uses a hidden text field to accept data coming from the \device. The \device emulates itself as a composite keyboard-mouse device when it is connected to the host. The \device emulates keystrokes that transmit encoded data to the \name JavaScript snippet that is sent to the remote server via \texttt{XMLHttpRequest} call.
\fi

%One instance of the communication channel is depicted in Figure~\ref{fig:keyExchange} that is used to establish a \tls channel between the \device and the remote server. The following section provides a detailed description of the key establishment protocol.




\subsection{\device Overlay of UI Elements}
\label{sec:systemDesign:transformation}

\begin{figure*}[t]
\centering
\includegraphics[trim={0 4.8cm 0 0}, clip, width=0.8\linewidth]{formTransform.pdf}
%\caption{\textbf{Transformation of UI elements: UI $\rightarrow$ QR code $\rightarrow$ \device generated UI overlay.} Automated transformation of the UI elements (\one) by the \name JavaScript snippets that detects the presence of the device. The corresponding \html source shows the UI elements that requires integrity/privacy protection. These UI elements are transformed into a QR code (\two). The QR code encodes a UI specification that recreates the transformed UI. Specification~\ref{snippet:UISpecification} shows the corresponding UI specification that is created by the \name \js code. The QR code is then decoded and overlaid (\three) on the HDMI stream by the \device. Upon the user's action on the overlaid UI elements, the device signs all the input data and send them to the remote server. As the rendered UI is generated and overlaid by the \device, it also ensures the integrity of the UI elements. Note that the intermediate QR code transformation (\two) is not visible by the user as it is decoded instantaneously by the device.}
\caption{\textbf{Transformation of UI elements: UI $\rightarrow$ encoded specification $\rightarrow$ \device generated UI overlay.} \one The actual webpage and the corresponding \html source shows the UI elements that requires integrity protection. \two These UI elements are transformed into an encoded UI specification (here by QR code that encodes a UI specification, e.g., Specification~\ref{snippet:UISpecification}) by the \name JS. The QR code. \three AThe QR code decoded and overlaid on the HDMI stream by the \device. \four Upon the user's action on the overlaid UI elements, the device signs all the input data. \five The \device sends these signed input data them to the remote server. Note, that the intermediate QR code transformation (\two) is not visible by the user as it is decoded instantaneously by the device.}
\label{fig:transformation}
\end{figure*}

%In the previous section, we describe the strawman solution using image/text analysis for IO integrity that has several drawbacks. In this section, we describe more efficient technique: transformation of the user interface (UI) elements.

We assume that the public certificate of the remote server is embedded into the \device. This allows the \device to verify any statement that is signed by the remote server's private key. It is also possible to establish a conventional \tls channel between the \device and the server using the HDMI channel. We provide the details how to build this \tls channel in Section~\ref{sec:systemDesign:confidentiality:tls}.

\subsubsection{\device generated UI overlay} \label{sec:systemDesign:transformation:overlay} 
This approach requires one extra component at the client side that is a JavaScript code snippet that we call \name JS. After the establishment of the secure channel between the \device and the remote server, the \name JS snippet that is served with the webpage transforms the UI elements that require IO integrity protection. Note that \name JS runs on the browser and is not trusted in our system model. We illustrate the method of UI transformation in Figure~\ref{fig:transformation}. The entire process has two phases: (i) UI $\rightarrow$ encoded specification generation, and (ii) Encoded specification $\rightarrow$ Overlay.




%%%old specification
% 
% { "formId": "form1",
%   "formName": "form1",
%   "domain": "secure-site.io"
%   "ui": [{ "id":"textbox_1",
%      "type":"textbox",
%      "label":"Sensitive field 1",
%      "text":"secret data 1"},
%    { "id":"textbox_2",
%      "type":"textbox",
%      "label":"Sensitive field 2",
%      "text":"secret data 2"},
%    {"id":"OK_button",
%      "type":"button",
%      "trigger":"true",
%      "label":"OK"},    
%    {"id":"Cancel_button",
%      "type":"button",
%      "trigger":"false",
%      "label":"Cancel"}]}
%%


\begin{mylist}
\item \textbf{UI $\rightarrow$ encoded specification generation.} In our implementation of \name, we use QR code as the intermediate representation of the UI elements that the \name JS generates and the \device interprets. Note that \name is agnostic of the type of intermediate representation. We use QR code as it is efficient and can be represented as an image in the HDMI stream that the \device intercepts. Figure~\ref{fig:transformation} shows this transformation between the step \one and \two. Note that the user never sees the QR code as it gets decoded and overlaid by the \device. The UI specification describes the UI elements that the \device interprets and use to generate the UI bitmap that is used for the overlay. One such concrete example of the UI specification is provided in Specification~\ref{snippet:UISpecification}. Section~\ref{sec:prototype:impl:qr} provides detailed implementation details of how the \name JS generates the encoded UI specification. 

\item \textbf{Encoded specification $\rightarrow$ Overlay.} Overlay is the next phase where the QR code that embeds the UI specification is interpreted by the \device and overlaid on the HDMI stream. The overlay faithfully recreates the UI to prevent any alteration in the user experience. The \device overlay is depicted in \three in Figure~\ref{fig:transformation}. The \device come with a small interpreter routine that converts the UI specifications to bitmaps that are then overlaid on the HDMI stream. The specification also contains the location and the size details (omitted from the Specification~\ref{snippet:UISpecification}). The \device uses this information to determine the specific UI element over which the user has her mouse pointer. As the \device parses all the HDMI frames for the QR code, the overlay does not get interrupted by scrolling.
\end{mylist}

\subsubsection{Integrity of the UI elements} Output integrity ensures the integrity of the UI elements that are sent by the remote server. As the specification sent by the server is signed (\texttt{signature} attribute in the UI specification mentioned in the Specification~\ref{snippet:UISpecification}), the \device-generated UI overlay ensures the integrity and the authenticity of the UI elements on the web page. Note that the \device always draws on top of the frames that are rendered by the compromised host. This way, the overlaid UI is always visible by the user on display and the host can not manipulate the overlays.

%\subsubsection{Integrity of the user input} After the UI elements are correctly overlaid on the screen, the users can interact with these UI elements. The user interaction with the overlaid UI element is no different then a standard UI, making the user habituation seamless. The UI specification encodes the behavior of all the generated UI elements, making the \device aware of the semantics of the UI objects. E.g., when a user selects a text box and types on her keyboard, the \device intercepts all the keyboard strokes and overlays the characters on the UI. When the user clicks on the \texttt{OK} button on the overlay, the device gathers all the intercepted keyboard and mouse events, signs them and send them to the remote server. One example of the proof-of-action is depicted in Figure~\ref{fig:transformation} that shows the payload that the \device sends to the server. Along with the keyboard data that is provided by the user, the \device also attaches the proof that the user indeed clicked on the \texttt{OK} button that leads to the action. Detailed description of how the input is recorded and committed is presented in Section~\ref{sec:systemDesign:commit}.


\subsection{Continuous Tracking of Mouse Pointer in the HDMI Frame}
\label{sec:systemDesign:analysis}


\begin{figure}[t]
\centering
\includegraphics[trim={0 5.8cm 8cm 0}, clip, width=\linewidth]{mouseAnalysis.pdf}
\caption{\textbf{Pointer integrity.} The device knows the initial pointer position by sending a high mouse value to the host that puts the pointer on a specific corner $(x_0, y_0)$. \one The \device captures the raw mouse events ($\Delta x, \Delta y$) from the mouse that is attached to the \device. \two The \device captures the frames from the HDMI channel and checks into the designated pixel position $(x_i + \Delta x, y_i + \Delta y)$ if there exists a pointer.}
\label{fig:mouseAnalysis}
\centering
\end{figure}

 
Protecting the integrity of the mouse actions (move, drag, and click), also known as the \emph{pointer integrity} is non-trivial as one needs to know the user's context (such as the location, acceleration of the mouse pointer) on the screen. \name extracts the user context by intercepting the HDMI frames from the host system. The \device sits between the host and the display device (the monitor) and captures the HDMI frame to extract the cursor context and overlay a \device-generated mouse pointer. The overlay provides a visual cue for the user about the correct location of the mouse pointer in the case the attacker wants to trick the user by spawning multiple pointers on the screen. Pointer integrity requires two steps: detection of the pointer from the HDMI frames and overlay of the \device-generated pointer.

\subsubsection{Detection of pointer} Figure~\ref{fig:mouseAnalysis} illustrates the high-level idea of the host system's HDMI frame analysis. To match the mouse polling rate with the display frame rate, the \device only queries the input device with the frequency of $60$ Hz. We assume that over the HDMI channel the host system sends frames at the rate of $60$ fps. We define the mouse trace as the time series $(\Delta x_i = |x_{t_{i-1}} - x_{t_{i}}|, \Delta y_i=|y_{t_{i-1}} - y_{t_{i}}|)$ delta co-ordinates: $\{(\Delta x_1, \Delta y_1), (\Delta x_2, \Delta y_2), \ldots, (\Delta x_n, \Delta y_n)\}$ from time $\{t_1, t_2, \ldots, t_n\}$. $x_{t_{i}}$ and $y_{t_{i}}$ denote the pixel position of the mouse pointer on the screen at time $t_i$. At the time of initialization, i.e., the first time the \device turned on after the host boot up, the \device pushes the mouse at the left-upper corner of the screen, hence the initial mouse pointer position is set to $(x_0, y_0) = (0, 0)$.  
Note that a mouse only provides displacement of over $x$ and $y$ coordinates. This corresponds that at time $t_i$ the mouse reported $(\Delta x_i, \Delta y_i)$ displacement to the \device. Assume that the frames coming from the host system to the \device are: $\{f_1, f_2, \ldots, f_n\}$ over the same time interval $(t_1, t_2,\ldots t_n)$. At time $t_i$, the \device looks into the frame $f_i$ and draws a square centered at $(x_i, y_i)$ with sides of length $X$ (which is enough to cover a mouse cursor on the screen\footnote{In our evaluation we saw that only a 30 $\times$ 30 square px. could cover the default mouse pointer provided by Ubuntu OS of resolution width x height.}). Then the \device checks if there exists a mouse inside this square or not. In case there exists a mouse cursor, the \device allows further user interactions; otherwise it stops all the communications and shows an error on display.

\subsubsection{Calibration} When the user connects the \device for the first time after booting up, the \device performs an automated calibration to find the pointer. The \device gives very high mouse movement values that pushes the mouse pointer to the top-right corner of the screen. Then the \device tries to find the pointer there from the HDMI frames. If the \device is successful in to finding the mouse pointer there, it continue tracking the ponter thereafter.  

\subsubsection{Overlay of the mouse pointer} The \device draws a mouse pointer that is visible by the user. The overlaid mouse pointer is on top of the host rendered mouse pointer. The overlay provides a telltale sign to the user about the location of the mouse pointer in the case the host renders other mouse pointers on the screen to confuse the user. To emphasize the location of the mouse pointer, the \device highlights the overlaid pointer by dimming the rest of the part of the screen when there is a threshold time break (i.e., no input coming from the user for around 3 seconds). Note that as long as the \device is connected to the host, it always detect and overlay the mouse pointer irrespective of the application that is running on the host. 


\subsubsection{Coping with the disappearing cursor} Many OS offers a feature where the mouse pointer disappear from the screen when the user types in a text editor/browser. When the user moves her mouse again, the cursor again appears at the exact same position from where it was disappeared in the first place. From the \device's perspective, it is hard to distinguish between this case and the attacker deliberately removing the mouse pointer from the screen. To handle the case where the mouse pointer disappears from the screen when the user starts typing, the \device listens to all the keyboard input as the keyboard is also connected to the \device. When the \device detects there are keystrokes, it expects the cursor to be disappeared from the screen. When the \device again starts receiving input from the mouse, it checks if the cursor appears again on the screen at the exact location from where it disappeared - this way the \device ensures the consistency of the pointer position.  


\subsubsection{Handling different mouse cursors} The \device is preloaded with the template images of the mouse cursor for identification. For our \name prototype implementation, we use the default cursors provided by the Ubuntu OS. This allows the \device to identify the cursor when it changes on the screen, e.g., from pointer to a hand when the user hovers her mouse over a link on the browser. 



\subsection{Secure Attention Sequence}
\label{sec:systemDesign:SAS}

Secure Attention Sequence (SAS) is a sequence of actions\footnote{Such as keystrokes \texttt{Ctrl+Alt+Del} that allows the user to provide her credential.} executed by the user that is completely trustworthy. SAS prevents an untrusted system from triggering an event that is otherwise sensitive to the user. Note that SAS is a well-researched topic in the context of UI/UX design. \name uses off-the-shelf SAS mechanism that provides a visual aid for the user to distinguish overlaid UI and the mouse pointer location and adapts it into the system. 

SAS eliminates any possibility that the attacker either i) creates any malicious UI in the screen that tricks the user into submitting sensitive information or provide fraudulent information on the screen that influences user, or ii) spawns a fake mouse pointer to trick the user into clicking into a different location, i.e., clickjacking. To protect against such scenarios, \device also provides SAS using which the user can highlight the overlaid UIs and the mouse pointer on display. 


\myparagraph{SAS policy} The remote server can set configurable SAS policy per overlaid UI (i.e., QR code). The SAS policy is defined in the \texttt{SAS} attribute in the example specification provided in Specification~\ref{snippet:UISpecification}. By default, the overlaid UI is locked from the user and requires a key press from the user to unlock the sensitive UI. This information is overlaid on the UI to remind the user to execute this. One example policy could be \texttt{Ctrl+d:5} which denotes that user needs to press key `\texttt{Ctrl+d}' to unlock the UI overlay. Pressing this key also trigger the \device to back out the HDMI frames except for the UI overlay and the mouse pointer overlay for a specified time (here for $5$ seconds). As the \device overlays on the HDMI frames, the attacker can not manipulate any overwriting by rendering anything on top of them. The dimming of the non-overlaid part of the screen is also activated when the user enters or exits the overlay. This aids the user in the case she is following a fake mouse pointer on the screen while the real cursor enters/exits the sensitive UI.

%\name provides a mechanism for SAS where the user forces the \device to highlight the overlaid UI and the mouse pointer location. SAS eliminates any possibility that the attacker either i) creates any malicious UI in the screen that tricks the user into submitting sensitive information, or ii) spawns a fake mouse pointer to trick the user into clicking into a different location, i.e., clickjacking. To protect against such scenarios, \device also provides SAS using which the user can highlight the overlaid UIs and the mouse pointer on display. 
%This can be activated explicitly by the user by pressing \texttt{ctrl} button.
%As the \device overlays on the HDMI frames, the attacker can not manipulate any overwriting by rendering anything on top of them. 
%There are two SAS mechanisms that \name employs. One is intrinsic to the user action. When the user moves her mouse, the \device briefly outlines the overlay UI. The other SAS mechanism is when the user presses a button on the \device, the \device dims the rest of the part of the display except the legitimate mouse pointer and the overlaid UIs. By doing so, the user can easily discriminate the secure UIs and the legitimate mouse pointer. 


%\subsubsection{Mouse pointer overlay} The mouse-pointer overlay also employs SAS. When the user press \texttt{ctrl} key on her keyboard, the \device dims all the portion of the screen except the overlaid UI and the overlaid mouse pointer. This allows the user to determine the legitimate pointer and the overlaid UI. For additional user attention, the mouse pointer also changes the out border color (\textcolor{red}{red} to \textcolor{green}{green}) when the user enters inside the overlaid UI. This also works as a indicator for the user if the browser/host impersonates a overlaid UI to trick the user into providing some security critical data.
 

\subsection{Transmit User Input}
\label{sec:systemDesign:commit}

When the user finishes providing her input over the input device (mouse and keyboard), the \device sends these values (with signature to ensure integrity) to the remote server. Sending these signed input values to the server requires a upstream channel from the \device to the server.

\subsubsection{Upstream channel}\label{sec:systemDesign:commit:upload} The \emph{upstream} channel, i.e., the data from the \device to the remote server is transmitted using the \name JavaScript snippet that is served by the remote web server. The \name JavaScript snippet uses a hidden text field to accept data coming from the \device. The \device emulates itself as a composite human interface device (HID) when it is connected to the host. The \device emulates keystrokes that transmit encoded data to the \name JavaScript snippet that is sent to the remote server via \texttt{XMLHttpRequest} call.

\subsubsection{Sending input data}\label{sec:systemDesign:commit:send}
The user input transmission procedure is illustrated in Figure~\ref{fig:transformation}. This has two phases: \emph{record} and \emph{transmit} as described in the following:

\begin{mylist}
\item \textbf{Record.} After the UI elements are correctly overlaid on the screen, the users can interact with these UI elements. The user interaction with the overlaid UI element is no different then a standard UI, making the user habituation seamless. The UI specification encodes the behavior of all the generated UI elements, making the \device aware of the semantics of the UI objects. E.g., when a user selects a text box and types on her keyboard, the \device intercepts all the keyboard strokes and overlays the characters on the UI.
When user enters input data in the rendered overlay UI elements (such as textbox, button, slider, radio button, etc.), the \device records that in a (key, value) pair where the key is the identifier of the UI element (\texttt{id} in Specification~\ref{snippet:UISpecification}) and the value is the user provided value. The \texttt{type} of the UI elements determine what information to record. For example, the \device records all the keystrokes when a textbox is selected, the value corresponding to the position of the slider is recorded when the user interacts with a slider, etc. One example of the recording of the input data corresponding to the UI illustrated in Figure~\ref{fig:transformation} and Specification~\ref{snippet:UISpecification} is: 
\begin{align*}
Record = & (textbox\_1, Data\_1);(textbox\_2,Data\_2)
\end{align*}

\item \textbf{Transmit.} In the transmit phase, the \device waits for the user to select UI element which has a \texttt{trigger} capability (see Section~\ref{sec:systemDesign:transformation:overlay}).  When user clicks the \texttt{OK} button, the device signs the record. One such signed packet is also illustrated in Figure~\ref{fig:transformation}. Upon clicking, the \device encodes the entire packet into a base64 encoded string and submit it to the \name JavaScript snippet (over the uplink channel by emulating itself as a keyboard device). The \name \js snippet then sends the packet to the remote server by a \texttt{XMLHttpRequest} call.
\end{mylist} 

\subsubsection{Server-side verification} \label{sec:systemDesign:commit:verification}Upon receiving the signed input data from \device, the remote accepts the input if the signature verification is successful. Note, if a input field is annotated as \texttt{protect=``true''}, the server does not accept any input with the the \device signature. This prevents the attacker-controlled host to submit data. 

\subsubsection{Changing browser tabs or browsers}
The \device supports multiple browsing tabs across multiple browsers. The UI specification contains \texttt{formId} and \texttt{domain} that works as the unique identifier for a specific form served from a specific web server. The \device can maintain multiple parallel TLS connection to web servers. Depending on the observed \texttt{formId} and \texttt{domain} (refer to Specification~\ref{snippet:UISpecification}), the device retrieved the data that is entered by the user. This way even if the user switches tabs, the \device can still allow editing the forms across tabs.


\subsection{Main Protocol}
\label{sec:systemDesign:mainProtocol}

\begin{figure}[t]
\centering
\includegraphics[trim={0 6.5cm 12cm 0}, clip, width=\linewidth]{systemDesign.pdf}
\caption{\textbf{Flow of the \name main protocol.} The figure shows the high-level protocol flow and the main messages that are exchanged between the remote server, host, \device, and the Io devices.}
\label{fig:systemDesign}
\centering
\end{figure}


In the previous Sections, we explain the basic components of \name. Now in this Section, we describe the flow of \name protocol by putting these components together. The outline of \name is illustrated in Figure~\ref{fig:systemDesign}. In the flow, we assume that the user already clicked on a web link or typed the URL in the address bar of the browser. This also allows the \device and the remote server to establish a \tls channel using the method described in Section~\ref{sec:systemDesign:confidentiality:tls}. The rest of the steps are the as the following:


\begin{mylist}
  \item[\one] The browser renders the webpage that comes with \name JS. As described in Section~\ref{sec:systemDesign:transformation}, \name JS transforms the UI elements to a QR code that contains the equivalent UI specification.
  \item[\two] The graphics driver sends the rendered frame to the \device over the HDMI channel.
  \item[\three] The \device intercepts the HDMI signal and decode the QR code to retrieve the UI specification. After the decoding of the QR code, the \device renders the bitmap corresponding to the specification as described in Section~\ref{sec:systemDesign:transformation:overlay}.
  \item[\four] The \device send the HDMI frame with the UI overlay to the display device.
  \item[\five] After observing the HDMI frame and the overlaid UI, the user passes her input to the \device via the keyboard/mouse that is connected to the \device over the \usb interface.
  \item[\six] The \device uses raw mouse data and the HDMI frames to interpolate the mouse pointer using the method described in Section~\ref{sec:systemDesign:analysis}. This ensures pointer integrity. \device also overlays a mouse pointer on the HDMI frames.
  \item[\seven] When the user input her data to the host, the \device records her input data (Record phase in Section~\ref{sec:systemDesign:commit:send}).
  \item[\eight] The \name JavaScript snippet also acts as a upstream channel from the \device to the remote server (refer to Section~\ref{sec:systemDesign:commit:upload}). Via this channel, the \device sends the signed user action to the remote server. This signed user action can be seen as the second factor for the integrity of the user input data. (Commit phase in Section~\ref{sec:systemDesign:commit:send})
  \item [\nine] The server verifies the data from the two channels that are submitted by the host and the \device. (Section~\ref{sec:systemDesign:commit:verification})
\end{mylist}



\iffalse
\begin{figure}[h]
\centering
\includegraphics[trim={0 11cm 16.5cm 0}, clip, width=\linewidth]{inputPrivacy.pdf}
\caption{Input Confidentiality}
\label{fig:inputPrivacy}
\centering
\end{figure}
\fi




\subsection{Input \& Output Confidentiality}
\label{sec:systemDesign:confidentiality}


In the previous sections (the main protocol is described in Section~\ref{sec:systemDesign:mainProtocol}), we describe how the \name \js and the \device together transform and overlay the UI to ensure the integrity of the UI and the input data. Minor modification can be made to this method to introduce confidentiality to all IO. One of the major components for achieving IO confidentiality is to establishing a secure channel between the server and the \device. Using this \tls channel, the remote server sends all the UI elements to the \device and the \device submits all user inputs to the remote server.

\subsubsection{Establishing \tls}
\label{sec:systemDesign:confidentiality:tls}

\begin{figure}[t]
\centering
\includegraphics[trim={0 8cm 18cm 0}, clip, width=0.8\linewidth]{keyExchange.pdf}
\caption{\textbf{Key establishment.} A snapshot of the key exchange web page that is used to communicate the public certificates of the device and the remote server. This page only lasts for a few milliseconds. Hence the page is practically invisible to the user. The QR-code displayed on the web page serves as the downstream channel from the remote server to the \device, whereas the text field is the upstream channel.}
\label{fig:keyExchange}
\centering
\end{figure} 


When the user opens up a webpage that supports \name mechanism, key exchange is the first step that is executed by the \device. Also, the key exchange phase is crucial as the remote server also decide if the user has a \device. We assume that the remote server already has the \device's certificate, or some offline registration takes place. An instance of the key exchange mechanism of \name is illustrated in Figure~\ref{fig:keyExchange}. The flow of the key exchange mechanism is as the following:

\begin{mylist}
  \item The remote web server serves the web page that shows a QR code that encodes the signed public key of the remote server (server hello in TLS). This page has a $5$ seconds timeout.
  \item The device captures the frames and looks for a QR code. As soon as the device finds one, the device decodes the QR code and verifies it.
  \item If the verification is successful, the device emulates itself as a keyboard device to the host system. The device then encodes its signed public certificate to hexadecimal and send it as a keystroke to the host (client hello in the TLS). For signature, the \device uses the root key of the device manufacturer.
  \item The \name  JavaScript snippet looks for the keystrokes, and as soon as it gets a string of a specific length, it sends the key strokes to the remote server, and the \name JavaScripts loads the webpage.
  \item In case the user does not have a \device, the step mentioned above does not take place within the $5$ seconds timeout period. In that case, \name JavaScript snippet concludes that the user does not have a \device. This allows the webpage to fallback to conventional web UIs that do not involve \device for their operation.
\end{mylist}

After this, both the device and the remote server have each other's public certificates. Using these certificates, both the \device and remote server calculates the shared secret using the authenticated Diffie-Hellman protocol~\footnote{Assume that $(g^x, x)$ and $(g^y, y)$ being the public-private key pair of the remote server and the \device respectively, where $g$ being a generator of a group $G$. The remote server sends a QR code that encodes a CA-signed $g^x$. The \device transmits signed $g^y$ to the remote server. Both the remote server and the \device computes $g^{xy}$ as the shared secret. Detailed description can be found  in~\cite{blake1998authenticated}.}.




\lstset{language=HTML, frame=tb, caption=\small{\textbf{HTML page from the remote server that contains the encrypted UI specification for IO confidentiality.}} , label = snippet:encryptedHTML, firstnumber =1}
\begin{figure}[t]
\small
\begin{lstlisting}[mathescape=true]
<form action="/some_action">
  Text box 1:<br>
  <input type="text" name="text_box_1">
  <br> text box 2:<br>
  <input type="text" name="text_box_2">
  <encrypted_qr><!-encrypted UI specification->
  0x4a5c4... </encrypted_qr>
  <script> [JS outputs QR code that encodes 
  encrypted specification] </script>
</form> 
\end{lstlisting} 
\end{figure}



\begin{figure}[t]
\centering
\includegraphics[trim={0 3cm 16.5cm 0}, clip, width=0.85\linewidth]{activityPrivacyRender.pdf}
\caption{\textbf{IO confidentiality.} The figure shows how \name achieves the confidentiality of the UI elements and the mouse pointer in the presence of a compromised host. The upper screenshot shows the host's view of the display while the lower one shows the user's view. The host can only see a QR code where the specification is encrypted by the \tls session key between the \device and the remote server. The user saw the decoded and overlaid UI objects that are retrieved from the QR code sent by the remote server (as described in Section~\ref{sec:systemDesign:transformation}).}
\label{fig:activityPrivacy}
\centering
\end{figure}

After the server and the \device establish the \tls channel, \name is ready to provide IO confidentiality.

\subsubsection{Output confidentiality} Output confidentiality ensures that no information sent from the remote server is visible to the host. To enable output confidentiality, the UI overlay mechanism that is described in Section~\ref{sec:systemDesign:transformation} is modified slightly. Here we \name does not required \name JS to transform all the UI elements to QR code specification. A small server-side module that is very similar to \name JS transforms the UI elements to the UI specification (one such specification is provided in Specification~\ref{snippet:UISpecification}) and encrypt the specification with the \tls session key (Section~\ref{sec:systemDesign:confidentiality:tls}). The \device decodes the QR code from the intercepted HDMI frames, decrypts the specification and overlays the UI on the HDMI frames. One example is provided in the HTML Snippet~\ref{snippet:encryptedHTML} where the \texttt{<encrypted>} tag contains the encrypted UI specification from the server. The \name JS (inside \texttt{<script>} tag) encodes this encrypted UI specification to a QR code.

\subsubsection{Input Confidentiality} When the user enters her mouse pointer into the overlaid UI boundary, the \device stops transmitting any mouse data to the output system, making the host completely oblivious to any mouse movement done by the user. Likewise when the user selects a UI element, for example a radio button that is shown in Figure~\ref{fig:activityPrivacy},the \device encrypts and signs the packets and sends the packet to the remote server making the input commands/values hidden from the attacker-controlled host.  




