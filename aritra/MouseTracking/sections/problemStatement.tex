\section{Problem Statement}
\label{sec:problemStatement}

\iffalse
% note: the problem in the current infrastructure: the computer has access to raw data from the server, and the user
A user communicates with a remote server through a host interface which gives %The user typically sends inputs to the server through a keyboard or a mouse and takes responses on a screen. 
the computer access to the plaintext data that goes from the user to the server and vice-versa. Thus, a compromised host can always steal or modify the information going on both directions. An adversary that compromises user's host is able to alter user intentions, i.e., it can perform arbitrary actions on behalf of the user. Such an adversary is very powerful and difficult to be prevented by a remote server. The consequences of such attacks are critical depending on the importance of the target application. The attacker can redirect user's payments to a malicious account, read all sensitive communication of the user, or even threaten individuals lives by misconfiguring a medical device. 

% note: full compromise is possible in many ways, so we need trusted channels
The complexity of nowadays computer systems does not restrict such attackers, but on the contrary it opens up new possibilities. The adversary can jeopardize user inputs or outputs by attacking the user's computer directly, e.g., Man-in-The-Browser (MiTB) attacks, keyloggers, or malwares that can escalate their privileges to gain root access on the system. Furthermore, many sensitive applications offer their services via a native application or a website that usually imports third-party libraries. An adversary can craft a malicious piece of code and add it on a popular open source library to attack user inputs on specific applications, known as software supply chain attacks. This kind of attack is already seen on the past for websites managing cryptocurrencies wallets~\cite{softSupplyChainAttack, jsSupplyChainAttack}. Therefore, the necessity of a trusted channel between the remote server and the user is critical in any sensitive application.

%To mitigate these attacks, in some specific applications, e.g., online payments, the remote server asks the user to confirm her action on a second device (phone or special hardware). However, this defense mechanism is possible on very limited use-cases and it only prevents the adversary from altering the user action, but the sensitive information still leaks. 

%note: inspiration and what do we need
The existing systems do not provide any interface to the remote server to learn if the user's machine is compromised or not. Trusted execution environments (TEEs) such as Intel SGX bring trust into the processor, effectively eliminating the need to trust more significant code base (motherboard, memory modules, OS and other applications). However, the problem of isolating user's input and output remains as the operating system handles the IO drivers and data. Therefore, service providers typically operate with the assumption that the data they receive are genuinely generated by honest users and not altered by a compromised computer. On a few use cases, e.g., online payments, the service providers ask for user confirmation on a second device (phone or special hardware) to prevent such attacks. However, this solution is limited to few applications and it only gives integrity (the attacker still learns all inputs) for the user inputs. Our goal is to design a solution that does not require significant changes into the current infrastructure and provides the following features: (i) input integrity and privacy, (ii) output integrity and privacy, (iii) sandboxed activity, and (iv) low-TCB.

%Security properties such as integrity and privacy protect user's data from an adversary who wants to either make arbitrary changes to the data or see the sensitive data. Trusted execution environments (TEEs) such as Intel SGX supports isolation execution environments that protect the sensitive application from a malicious application. Privacy of the application data is achieved through memory encryption that hides the sensitive user/application data from the operating system. Remote attestation ensures that the target platform runs the proper version of an enclave. There also exists other TEEs that provides a similar set of features. Technologies such as Intel SGX bring trust into the processor, effectively eliminating the need to trust more significant code base (motherboard, memory modules, OS and other applications). But the problem of isolating user's input and output remains as the operating system handles the IO drivers and data. There exists some solution of building a trusted path between the user and the protected applications (enclaves in the SGX terminology), none of the existing methods providers a robust and generic solution that requires very few changes into the current and legacy system.

%In this paper, we tackle the problem of securing sensitive user IO data from arbitrary IO devices. We assume that the attacker compromises host systems that include the hardware, the operating systems, and all the installed applications. Such an attacker model allows it not only to steal the sensitive information from the user but also alter them. Additionally, as the attacker is in full control of the host system, he can alter the UI elements on the screen, change the location of the pointer, modify the speed and the acceleration of the pointer, etc. Such a capability of the attacker makes the problem of protecting the integrity of the mouse pointer a challenging one. Even though the problem of preserving the integrity/privacy of the keyboard input is addressed in the recent literature, integrity/privacy for mouse pointer is still an unsolved problem. Furthermore, the user is completely unprotected, she cannot verify that her inputs are transferred to the server securely, or even have the guarantee that she is communicating with the legitimate server. 

%In other words, in this paper, we focus on a new way to build systems that bring a user's trust into IO devices and data - keeping in mind that the solution works by almost no or little changes to the existing systems. Additionally, we also assume that the user's platform may nor support any TEEs. Such assumption reduces the trusted code base to bare minimal and making the problem of protecting user IO data under strong adversary model a challenging task.      
\fi


In this section, we motivate our work, explain the security properties, describe exiting literature lacks proper solution and explain the goals of our work.

\subsection{Motivation: Trusted Path}


A user communicates with a remote server through a computer interface, we call it as the host system, that which gives the host access to the plaintext data that goes from the user to the server and vice-versa. Thus, a compromised host can always steal or modify the information going on both directions. An adversary that compromises user's host is able to alter user intentions, i.e., it can perform arbitrary actions on behalf of the user. Such an adversary is very powerful and difficult to be prevented by a remote server. The consequences of such attacks are critical depending on the importance of the target application. The attacker can redirect user's payments to a malicious account, read all sensitive communication of the user, or even threaten individuals lives by misconfiguring a medical device. 


The complexity of nowadays computer systems does not restrict such attackers, but on the contrary it opens up new possibilities. The adversary can jeopardize user inputs or outputs by attacking the user's computer directly, e.g., Man-in-The-Browser (MiTB) attacks, keyloggers, or malwares that can escalate their privileges to gain root access on the system. Furthermore, many sensitive applications offer their services via a native application or a website that usually imports third-party libraries. An adversary can craft a malicious piece of code and add it on a popular open source library to attack user inputs on specific applications, known as software supply chain attacks. This kind of attack is already seen on the past for websites managing cryptocurrencies wallets~\cite{softSupplyChainAttack, jsSupplyChainAttack}. Therefore, the necessity of a trusted channel between the remote server and the user is critical in any sensitive application.

%To mitigate these attacks, in some specific applications, e.g., online payments, the remote server asks the user to confirm her action on a second device (phone or special hardware). However, this defense mechanism is possible on very limited use-cases and it only prevents the adversary from altering the user action, but the sensitive information still leaks. 

%note: inspiration and what do we need
The existing systems do not provide any interface to the remote server to learn if the user's machine is compromised or not. Trusted execution environments (TEEs) such as Intel SGX bring trust into the processor, effectively eliminating the need to trust more significant code base (motherboard, memory modules, OS and other applications). However, the problem of isolating user's input and output remains as the operating system handles the IO drivers and data. Therefore, service providers typically operate with the assumption that the data they receive are genuinely generated by honest users and not altered by a compromised computer. On a few use cases, e.g., online payments, the service providers ask for user confirmation on a second device (phone or special hardware) to prevent such attacks. However, this solution is limited to few applications and it only gives integrity (the attacker still learns all inputs) for the user inputs. 
%Our goal is to design a solution that does not require significant changes into the current infrastructure and provides the following features: (i) input integrity and privacy, (ii) output integrity and privacy, (iii) sandboxed activity, and (iv) low-TCB.

\begin{figure}[t]
\centering
\includegraphics[trim={0 1cm 10cm 0}, clip, width=\linewidth]{motivation.pdf}
\caption{\textbf{Motivating examples.} 1) Pointer based UI elements that sets parameters to remote safety-critical device, 2) E-voting where the voting privacy and integrity is critical, 3) Financial transactions such as bitcoin wallet that shows sensitive information such as the user's private key and 4) web applications that provide an option for the user to reveal credentials.}
\label{fig:motivation}
\centering
\end{figure}

\subsection{Fine-grained Security Properties}


As we discussed in the previous section, \emph{Trusted path} provides the security properties (privacy and integrity) to the IO data between the users and the end systems. In principle trusted path solves the general security problem of the IO data. But practically establishing a trusted path in general IO devices is a nontrivial problem specifically if one considers the plethora of complex UI objects and input methods.

We list the fine-grained security properties that we want to provide in our proposed solution. We now discuss these security properties and corresponds them with the motivating example.

\begin{mylist}
  \item \textbf{Input integrity and privacy.} These properties define that any input that is coming from the user input devices are fully protected in two ways: i) the inputs are that issued by the user reaches to the remote end-point as it was meant to be - \emph{integrity}, and ii) in specific application scenarios, the attacker-controlled host is entirely oblivious about the input from the user - \emph{privacy}. We can further fine grain the user input to the granularity of user-issued \emph{commands} and user-issued \emph{activities}. Both of which are generated by the input devices from the user. User-issued commands are the specific actions that are executed by the user. Commands are transmitted to the remote end-system, e.g., filling a text-box with a parameter and clicking the \emph{submit} button issues a \texttt{HTTP} request to the remote server. The user-issued activities are actions that are given to the host system which are not directly transmitted to the remote end-system but causes user-issued commands, e.g., mouse movement from one point to a button that user intends to click.  
  
  \item \textbf{Output integrity and privacy.} Similar to the input integrity, output integrity ensures that information that is sent by the remote endpoint reaches to the user as the way it was meant to be. We further fine-grain output integrity and privacy by dividing it into two parts: \emph{data} and \emph{UI}. Data denotes the information that is passed from a remote entity to the host and UI denotes to the user interface structure that is presented to the user.
  
  \end{mylist}

\begin{figure}[t]
\centering
\includegraphics[trim={0 14cm 17cm 0}, clip, width=0.9\linewidth]{systemModel.pdf}
\caption{\textbf{Trusted path system model.} The figures shows the system and the attacker model of the trusted path.}
\label{fig:systemModel}
\centering
\end{figure}

\iffalse
\myparagraph{Advantages}

\begin{enumerate}
  \item The \device does not need to know the formatting/template of the page. As the \device only looks to the current mouse position, the structure of the page is somewhat irrelevant (?).
\end{enumerate}
\fi

\iffalse
\begin{table}[t]
\small
\centering
  \begin{tabular}{ l | c | c }
    \hline
     & TEE & no TEE \\ \hline
    \multirow{6}{*}{Hypervisor-based} & \multirow{6}{*}{SGX IO~\cite{weiser2017sgxio}} & Overshadow~\cite{Overshadow} \\ 
    & & Virtual ghost~\cite{criswell2014virtual}\\ 
    & & Inktag~\cite{hofmann2013inktag}\\ 
    & & TrustVisor~\cite{mccune2010trustvisor} \\ 
    & & Splitting interfaces~\cite{ta2006splitting}\\ 
    & & $SP^3$~\cite{yang2008using}\\ \hline
   Isolated execution & BASTION-SGX~\cite{BASTION-SGX} & Slice~\cite{azab2011sice}\\ 
   of APIs/Drivers & TrustOTP~\cite{sun2015trustotp} & CARMA~\cite{vasudevan2012carma} \\ \hline
   External trusted  &  \multirow{2}{*}{Fidelius~\cite{Fidelius}} & IntegriKey~\cite{IntegriKey} \\
   hardware based &  & FPGA-based~\cite{brandon2017trusted} \\
    %&  & \textcolor{blue}{Our Solution (I + O + Activity)} \\
    \hline
    Handles both & \multirow{2}{*}{\textcolor{red}{None}} & \multirow{2}{*}{\textcolor{blue}{Our Solution}} \\
    keyboard + mouse &  & \\
    \hline
  \end{tabular}
  \caption{Summarization of existing trusted path solutions. Note that in the table, switching systems from left to right or top to bottom, reduces the trust assumption. For example, TEE based solutions, such as SGX-based trusted path solution requires trust on the physical processor packages, SGX APIs, quoting and launch enclaves and Intel attestation service.}
\end{table}
\fi


\begin{table*}[t]
\small
\centering
  \begin{tabular}{ l || c | c | c | c || c | c | c | c}   
   & \multicolumn{4}{c||}{Trust Assumption} & \multicolumn{4}{c}{Features}\\  \hline
   & Requires TEE & Hypervisor & Isolated API/Drivers & External trusted HW & Keyboard & Pointer & Display & Activity \\
    \hline
    SGX IO~\cite{weiser2017sgxio} 				& \yes 	& \yes & \yesNope & \no & \yes & \no & \no & \no\\
    Overshadow~\cite{Overshadow} 				& \no 	& \yes & \no & \no & \yes & \no & \no & \no\\
    Virtual ghost~\cite{criswell2014virtual} 	& \no 	& \yes & \no & \no & \yes & \no & \no & \no\\
    TrustVisor~\cite{mccune2010trustvisor} 		& \no 	& \yes & \no & \no & \yes & \no & \no & \no\\
    Inktag~\cite{hofmann2013inktag} & \no 	& \yes & \no & \no & \yes & \no & \no & \no\\\
    Splitting interfaces~\cite{ta2006splitting} & \no 	& \yes & \no & \no & \yes & \no & \no & \no\\
    $SP^3$~\cite{yang2008using} 				& \no 	& \yes & \no & \no & \yes & \no & \no & \no\\
    BASTION-SGX~\cite{BASTION-SGX}			    & \yes 	& \no & \no & \no & \yes & \no & \no & \no\\
    Slice~\cite{azab2011sice}				    & \no 	& \no & \no & \no & \yes & \no & \no & \no\\
    TrustOTP~\cite{sun2015trustotp}			    & \yes 	& \no & \no & \no & \yes & \no & \no & \no\\
    Fidelius~\cite{Fidelius}			   	    & \yes 	& \no & \yes & \yes & \yes & \no & \yesNope & \no\\
    FPGA-based~\cite{brandon2017trusted}		& \yes 	& \no & \yes & \yes & \yes & \no & \yesNope & \no\\
    IntegriKey~\cite{IntegriKey}				& \yes 	& \no & \yes & \yes & \yes & \no & \yesNope & \no\\
    CARMA~\cite{vasudevan2012carma}			    & \no 	& \no & \no & \no & \yes & \no & \no & \no\\ \hline
	\textbf{\name}	    		& \no 	& \no  & \no & \yes & \yes & \yes & \yes & \yes\\
    \hline
  \end{tabular}
  \caption{Summarization of existing trusted path solutions.}
\end{table*}




\subsection{Potential Solutions and their drawbacks}

Given the attacker's model, there exist several solutions that solve the problem of a trusted path to and from the IO devices in the presence of a compromised host. But all of these solutions targeted for different problem settings and models:

%\subsubsection{Strawman solution: trusted OS} Trusted OS could be seen as a straightforward solution as traditionally OS handles all the IO drives, computation, and network communications. Assuming a trusted OS significantly reduces the complexity of the solution but the security assumption suffers. Trusted OS introduces a large trusted code base and a multitude of vulnerabilities. 

\subsubsection{Hypervisor-based solutions} Trusted hypervisors and secure micro-kernels are also choices for contrasting Trusted path. Sel4~\cite{klein2009sel4} is a functional hypervisor that is formally verified and has a kernel size of only $8400$ lines of code. In work done by Zhou et al.~\cite{zhou2012building}, the authors proposed a generic trusted path on $x86$ systems in pure hypervisor-based design. Examples of other hypervisor-based works can be found in systems such as Overshadow~\cite{Overshadow}, Virtual ghost~\cite{criswell2014virtual}, Inktag~\cite{hofmann2013inktag}, TrustVisor~\cite{mccune2010trustvisor}, Splitting interfaces~\cite{ta2006splitting}, $SP^3$~\cite{yang2008using}, etc.

\myparagraph{Trusted Execution Environments} TEEs are other ways to implement a trusted path between the IO devices and the users. Several TEEs such as Intel SGX, ARM TrustZone, TPM, Intel TXT, etc. can be used to achieve such functionality. Previous research works such as Intel SGX and trusted hypervisor-based SGXIO~\cite{weiser2017sgxio}, Intel SGX based ProximiTEE~\cite{dhar2018proximitee}, TPM and TXT based trusted path~\cite{filyanov2011uni}, and ARM TrustZone based trusted path~\cite{filyanov2011uni,sun2015trustotp} are the example of trusted path construction based on TEEs. All of these solutions require specialized platforms with processors that support such infrastructure. In our proposed solution, we concentrate on the non-specialized hardware platform where compatible TEE technologies may not be available.
Moreover, TEE requires trust assumption on the processors and additional code bases. One such example is INtel SGX where the trust model includes the physical processor package, SGX SDK, quoting enclave, launch enclave and Intel attestation service. Our proposed solution avoids such extensive trust assumptions and assumes that the entire platform is in control of the attacker.


\subsubsection{Dedicated hardware-based solution} Previous research works such as IntegriKey~\cite{IntegriKey} uses a low-TCB embedded device to introduce a second factor for input integrity. Fidelius~\cite{Fidelius} uses raspberry pi's and Intel SGX to create a secure channel between the keyboard and the display device. By doing so, Fidelius provides secure input and display for the character-based device - keyboard. In their work, Brandon et al. ~\cite{brandon2017trusted} demonstrate screen overlay on Android devices using FPGAs.

Note that majority of the previous works achieve some form of trusted path specifically for keyboard-based input. However supporting mouse and touch-based input, complex and generic user interfaces and protected user activity in a privacy-sensitive application is not a trivial task. Without proper analysis of every frame that the host system produces, it is not possible to track user intention. In our knowledge, our proposed solution is the first to provide such security properties including the activity privacy. Moreover, we want to achieve this in the absence of any TEE as the trust model of our scenario is significantly different. 



\subsection{Goals}

In the above, we discussed several solutions and research works that solve the problem of constructing a trusted path in different ways. What makes our solution different from them is the specific settings that it targets. We do not assume trust in the OS, particular drivers, TEEs, hypervisors, etc. As all of these techniques as mentioned earlier requires trust on a large code base. Instead, our proposed solution solves the problem of building an efficient, trusted path using an off-the-shelf microcontroller and single board computers that require a bare-minimum trust assumption. The most distinguishing factor is the type of IOs our proposed system targets to protect, namely mouse/touch inputs and sophisticated user interfaces. Existing research works primarily targets character-based input devices such as keyboards, and the methods cannot be trivially applied to mouse/touch-based inputs or complex UI elements. 

The goals are the following:

\begin{mylist}
  \item  \textbf{Pinter-based input devices.} Existing trusted path solutions primarily focus on keyboard beds input. Note that keyboard based input only considers text boxes, thus completely ignores other types of UI elements. Extending the protection mechanisms from keyboard to pointer is non-trivial as the same mechanism does not apply. The pointer interacts with many complex UI elements. As the attacker can manipulate with such UI elements, spawn or changes the location of the pointer, manipulate with the speed and acceleration of the pointer, etc., protecting the integrity and the privacy of the pointer data is a challenging task. The primary goal of this paper is to ensure the security of the pointer in a trusted path.
  
   
  % \item \textbf{Security.} Our proposed solution provides IO security and sandboxed activity to protect user IO data. Such implies that while deployed, the device always protects the integrity of user input and in specific application scenarios, it provides IO privacy and integrity. Additionally, a goal of the system is to also protect the integrity of the UI elements on the screen so that the compromised OS cannot modify it.
   
   \item \textbf{Activity privacy.}
   
  \item \textbf{Easy deployment and integration.} Our proposed solution focuses on deploying it to an existing system without any significant changes. This also ensures minimal or no changes to the user interaction with the systems.

\end{mylist}


