\section{Security Analysis}
\label{sec:securityAnalysis}


Next, what follows are the informal security analysis of the integrity and confidential properties of \name. 


\subsection{Integrity}
\label{sec:securityAnalysis:integrity}



\myparagraph{Changing IO signals} As only the \device can interact with the overlaid UI, the attacker can not manipulate the IO signals inside the overlaid UI. Moreover, the attacker can not generate and submit input data to the remote server as the server requires all the input data to be signed by the \device.

\myparagraph{Early form submission} This attack is not possible as all the input (both mouse and keyboard) are connected to the \device and only the \device can interact with the overlaid UI. This makes it impossible for the attacker to emulate a click on the overlaid part of the screen.  

\myparagraph{Attack on the mouse pointer tracking and overlay} 
The attacker may try to defeat the \name pointer tracking and overlay mechanism described in Section~\ref{sec:systemDesign:analysis} by introducing a malicious pointer that is visually more appealing to the user. Note that the \device overlaid mouse pointer is prominent and hard to miss. One can visualize it as an arms race between the attacker and the \device to grab the attention of the user. But, we can argue that this is a suboptimal strategy for the attacker as both of the pointers will be visible on the screen that can cause suspicion to the user. Also, when the real mouse pointer enters the overlaid area, the untrusted part, including the malicious mouse pointer, will be hidden by the focusing mechanism. Hence, we can conclude that executing clickjacking-like attacks is not possible in \name.



\myparagraph{Replay attack} All the sensitive UI elements from the remote server contain freshness value (nonce) alongside the signature. Hence, the replay attack is not possible.

\myparagraph{Not rendering QR code} The host may deny sending the QR code over the HDMI channel. This is considered as a denial of service, which is acceptable in our threat model as long as the user interacts with a compromised host. 
In any case, the host is not able to alter any user input or access encrypted data when \name with the confidential feature is required.
%However, the \device cannot observe and decode the encoded UI. As the remote server only accepts signed inputs from the \device corresponding to the sensitive UI, the host can not submit arbitrary input values to the remote server.

\myparagraph{Redirection} The attacker could redirect the user to a phishing website that renders visually identical UI to that of the legitimate website. Redirection attack cannot break the integrity of the input because a legitimate remote server always requires the signed input from the user. Without a valid signed QR code, the \device never signs the input value. 

\myparagraph{Malicious instruction on the screen} The attacker may put malicious instruction/labels on the untrusted part of the screen to influence user inputs. However, when the user enters inside the overlaid UI to send inputs, the focusing mechanism highlights only the secure UI and hides the rest of the screen. 
%The user attention focusing mechanisms enable the user to distinguish the trusted part of the screen from the untrusted part.

\subsection{Confidentiality}

\myparagraph{Redirection} The attacker could redirect the user to a phishing website that renders visually identical UI to that of the legitimate website. Redirection may compromise only the confidentiality of user inputs if the user does not trigger the SAS mechanism. Note that the \device can contain a whitelist of the remote server addresses and their corresponding certificates. The \device is only activated when it sees UI coming from the whitelist addresses.
%as the confidentiality of inputs requires the user to manually trigger the SAS to detect any sensitive UI elements that are overlaid by the \device.

\myparagraph{Side-channel leakages} Even though, the \device ensures that no mouse, keyboard events arrive at the untrusted host when the user executes some operation over the overlaid UI, one can not rule out all side-channel leakages. Depending on the application, the amount of time that the user spends or the entry/exit position of the mouse pointer may reveal some information to the attacker. 
\device could allow the remote server to specify additional policies in the specification to prevent side-channel attacks, e.g., a minimum amount of time that the device should not forward any event to the host after the user enters the overlay. We leave as future work defining such policies and implementing them on \name.
%However, for fixed length inputs such as the pin codes or credit card details, do not leak any information about the input.

\myparagraph{Mode Switching} The host could remove the QR code when the user is typing confidential data in the sensitive form. Absence of the QR code makes the \device to assume that the secure session has ended and the \device forwards the plaintext keystrokes and mouse movement to the host. To prevent the leakage of the input data, the \device continue to overlay and operate on the overlay till the user clicks submit or cancel (or any UI element that has a \texttt{trigger}  capability). This way, the attacker locks the UI from the attacker until the user finishes her session.

\subsection{Attacks toward \device} 

In \name trust model, we assume that the \device is trusted. However, in real-world, embedded systems are often vulnerable to attacks as the attacker can use the connection interfaces to reprogram the \device. In our case, the \device has only two interfaces with the host. The HDMI controller on the \device does not support any bidirectional data channel. However, the attacker could forge a QR code send it to the \device over the HDMI channel that may exploit the \device. As the code base of the \device is small, we assume that the code can be verified formally to protect against such attacks. The \usb interface on the \device is only unidirectional (\device $\rightarrow$ host) as the \device emulates itself as an HID. Hence the attacker also cannot exploit the \usb interface.  
