\section{Security Analysis}
\label{sec:securityAnalysis}

%In this section, we analyze the security of \name by modeling the interaction between the user, host, and the remote server.


%\subsection{Modelling the Protocol}
%\label{sec:securityAnalysis:modelling}

%\subsubsection{Modelling the user behavior}
%\label{sec:securityAnalysis:modelling:user}

\myparagraph{Modelling the user behavior} Correctly understanding the user behavior is critical as \name focuses on securing user interaction with the remote server. In Section~\ref{sec:systemDesign:userAttention}, we explain secure attention sequence (see SAS in Section~\ref{sec:confidentiality}) that \name uses to provide visual clues to the user. SAS enables use to i) follow the legitimate mouse pointer, and ii) recognize which part of the screen is overlaid by the \device. This is achieved by dimming out the screen except for the pointer and the UI overlay. We assume that the user behaves reasonably, hence the SAS mechanism is sufficient to isolate the trusted part of the screen (\device overlay) from the untrusted part of the screen. We also agree that system like \name may not achieve absolute security as the user may choose to ignore all visual cues made by the \device and choose to follow the untrusted notification from the host. Therefore, as long as the user does not get influenced by the information provided in the untrusted part of the screen and modifies her input, the \name remains secure.
\red{$\leftarrow$ revise.}
 %The formal security analysis is provided in Appendix~\ref{appendix:security}.


\subsection{Integrity}
\label{sec:securityAnalysis:integrity}

%From the theorem that is described in Section~\ref{sec:securityAnalysis:modelling}, we can prove that \name provide input integrity. The UI overlay ensures that the user has output integrity, i.e., all the sensitive information sent by the server cannot be manipulated by the host. Also, the \device signs all the input provided by the user. From Theorem~\ref{theorem:th3}, we can conclude that \name provides IO integrity as long as we assume that the user does not get influenced by any information shown in the non-overlaid part of the screen (Discussed in Section~\ref{sec:securityAnalysis:modelling:user}). \name additionally protects against replay attack as the form also contains nonce to ensure freshness.  

In Appendix~\ref{appendix:security:protocol}, we model \name as a finite state machine (FSM) as seen in Figure~\ref{fig:fsm}. Definition~\ref{def:inputIntegrity} and~\ref{def:outputIntegrity} provides the formal definition of input and output integrity respectively. In Theorem~\ref{theorem:th1}, ~\ref{theorem:th2}. and ~\ref{theorem:th3} we prove that as long as user reports everything that she sees on display to the remote server alongside her input over an authenticated channel, input integrity could be achieved. \name ensures output integrity the \device overlays the sensitive UI on the screen. By doing so, \name also ensures input integrity. 
\red{$\leftarrow$ revise.}

\myparagraph{Changing IO signals} As only the \device can interact with the overlaid UI, the attacker can not manipulate the IO signals inside the overlaid UI. Moreover, the attacker can not generate and submit input data to the remote server as the server requires all the input data to be signed by the \device.

\myparagraph{Attack on the mouse pointer tracking and overlay} 
The attacker may try to defeat the \name pointer tracking and overlay mechanism described in Section~\ref{sec:systemDesign:analysis} by introducing a malicious pointer that is visually more appealing to the user. Note that the \device overlaid mouse pointer is prominent and hard to miss. One can visualize it as an arms race between the attacker and the \device to grab the attention of the user. But, we can argue that this is a suboptimal strategy for the attacker as both of the pointers will be visible on the screen that can cause suspicion to the user. Also, when the real pointer enters the overlaid area, the untrusted part including the malicious pointer will be hidden by the focusing mechanism. Hence, we can conclude that executing clickjacking-like attacks is not possible in \name.

\myparagraph{Early form submission} This attack is not possible as all the input (both mouse and keyboard) are connected to the \device and only the \device can interact with the overlaid UI. This makes it impossible for the attacker to emulate a click on the overlaid part of the screen.  

\myparagraph{Replay attack} All the sensitive UI elements from the remote server contain freshness value (nonce) alongside the signature. Hence, the replay attack is not possible.

\myparagraph{Not rendering QR code} The host may deny sending the QR code over the HDMI channel. This is considered as a denial of service which is acceptable in our threat model as long as the user interacts with a compromised host. 
In any case, the host is not able to alter any user input or access encrypted data when \name with confidential feature is required.
%However, the \device can not observe and decode the encoded UI. As the remote server only accepts signed inputs from the \device corresponding to the sensitive UI, the host can not submit arbitrary input values to the remote server.

\myparagraph{Malicious instruction on the screen} The attacker may put malicious instruction/labels on the untrusted part of the screen to influence user inputs. However, when the user enters inside the overlaid UI to send inputs, the focusing mechanism highlights only the secure UI and hides the rest of the screen. 
%The user attention focusing mechanisms enable the user to distinguish the trusted part of the screen from the untrusted part.

\subsection{Confidentiality}

\myparagraph{Redirection} The attacker could redirect the user to a phishing website that renders visually identical UI to that of the legitimate website. Redirection attack cannot break the integrity of the input because a legitimate remote server always requires the signed input from the user. Without a valid signed QR code, the \device never signs the input value. 

The remote server also delivers confidential information only after creating a \tls channel with the user's device. Therefore, if the attacker redirects the user to a malicious page, the legit server does not deliver any information. On the other hand, redirection may compromise only the confidentiality of user inputs if the user does not trigger the SAS mechanism. 
%as the confidentiality of inputs requires the user to manually trigger the SAS to detect any sensitive UI elements that are overlaid by the \device.

\myparagraph{Side-channel leakages} Even though, the \device ensures that no mouse, keyboard events arrive to the untrusted host when the user executes some operation over the overlaid UI, one can not rule out side-channel leakages. Depending on the application, the amount of time that the user spends or the entry/exit position of the mouse pointer may reveal some information attacker. 
\device could allow the remote server to specify additional policies in the specification to prevent side-channel attacks, e.g., minimum amout of time that the device should not forward any event to the host after the user enters the overlay. We leave as future work defining such policies and implementing them on \name.
%However, for fixed length inputs such as the pin codes or credit card details do not leak any information about the input.




\subsection{Attacks toward \device}


\iffalse
\subsection{Protection against phishing attacks}
\subsection{Keyboard Manipulation Attacks and Defenses}
\subsubsection{Change user selected values}


\subsection{Mouse Manipulation Attacks and Defenses}
\subsubsection{Changing mouse position}
Changing the mouse position can be detected by the device as the device expects to find it in the location that the user provides. 
\subsubsection{Removing the mouse completely}
This is detectable by the \device as the \device no longer finds the mouse pointer in the screen at the designated position.  
\subsubsection{Add mouse cursor to confuse users}

\subsection{UI Manipulation Attacks and Defenses}
\subsubsection{Manipulate the position of the UI elements on the screen}
\fi