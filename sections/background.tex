\section{Background: Secure Web UI}
\label{sec:background}

The modern web browser provides a comprehensive set of mechanisms that protect users from malicious websites and \js. W3C UI security specification~\cite{w3c_spec} dictates such mechanisms. One known attack is hidden element attack where the malicious \js renders a button (could be a Facebook like) on top of another(could be a link to a malicious website). This tricks the user into clicking on the overlaid button while the user redirects to the malicious website. There exist several security policy directives. E.g., \texttt{input-protection} directive enforces several input protection heuristic to protect from such attacks.  \emph{Obstruction check} enable the browser to take screenshots of the screen area and to check if there is some overlaid element. Another heuristic is \emph{timing attacks countermeasure} where the browser maintains a list called \texttt{Display Change List} that contains all the UI changes on the DOM tree. Using this list, the browser checks if there is any UI change on the security sensitive UIs. The developers can define \texttt{input-protection-clip} in the HTML the defines a rectangular screen area whose intersection with the bounding rectangle of the whole document's body should be used as the reference area in the screenshot comparison.

InContext~\cite{huang2012clickjacking} presents different clickjacking attacks variants and their solution by ensuring context (both temporal and visual) and pointer integrity. In a clickjacking attack, a malicious \js renders a legitimate looking cursor on the browser. This fake cursor tricks the user into following that while the real cursor is on a sensitive UI element. One possible way to mitigate this attack is to focus user attention when the actual cursor is on a security-sensitive UI. Focusing user attention could be done by several means. Lightbox mechanism allows the browser to gray out the rest of the part of the screen except the security-sensitive UI element when the real cursor enters the UI. Freezing makes the portion of the frame suspended when the user enters the UI. %The browser also enforces a time delay of clicking on a sensitive button to allow a cool-down period.  
\red{Let's say more clearly that countermeasures above are from the paper, not our proposals.}