\section{Background: Secure Web UI}
\label{sec:background}

The modern web browser provides a comprehensive set of mechanisms that protect users from malicious websites and \js, assuming that both the browser and the OS are trusted. W3C UI security specification \cite{w3c_spec} dictates such mechanisms. One known attack is the hidden element attack where the malicious \js renders an button-like object (could be a Facebook like) on top of another (could be a link to a malicious website) \red{where the former is transparent to click}. This tricks the user into clicking on the overlaid button, but the user is redirected to a malicious website. There exist several security policy directives to prevent such attacks. For example, \texttt{input-protection} directive enforces several input protection heuristics to protect users. \emph{Obstruction check} enables the browser to take screenshots of the screen area and to check if there is any overlaid element. Another heuristic is \emph{timing attacks countermeasure} where the browser maintains a list called \texttt{Display Change List} that contains all UI changes on the DOM tree. Using this list, the browser checks if there is any UI change on the security sensitive UIs. Developers can define \texttt{input-protection-clip} in the HTML which defines a rectangular screen area whose intersection with the bounding rectangle of the whole document's body should be used as the reference area in the screenshot comparison.

InContext~\cite{huang2012clickjacking} presents different clickjacking attacks variants and possible solutions to ensure context (both temporal and visual) and pointer integrity. In a clickjacking attack, a malicious \js renders a legitimate looking cursor on the browser. This fake cursor tricks the user into following that while the real cursor is on a sensitive UI element. InContext proposes a way to mitigate this attack by attracting the user attention when the actual cursor is on a security-sensitive UI. Focusing user attention could be done by several means. Lightbox mechanism allows the browser to gray out the rest of the part of the screen except the security-sensitive UI element when the real cursor enters the UI. Freezing makes the portion of the frame suspended when the user enters the sensitive UI. %The browser also enforces a time delay of clicking on a sensitive button to allow a cool-down period.  