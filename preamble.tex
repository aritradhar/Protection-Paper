%%%%%%%%%%%%%%%%%%%%%%
%usenix
%\documentclass[letterpaper,twocolumn,10pt]{article}
%\usepackage{usenix2019,epsfig,endnotes}
%%%%%%%%%%%%%%%%%%%%%%
%CCS
\documentclass[sigconf, anonymous]{acmart}

% remove reference format for submission
\settopmatter{printacmref=false}

% remove copyright box for submission
\renewcommand\footnotetextcopyrightpermission[1]{}

% control headers 
\fancyhf{} 
\fancyhead[C]{Anonymous submission \#9999 to ACM CCS 2019} % TODO: replace 9999 with your paper number
\fancyfoot[C]{\thepage}

\setcopyright{none} % No copyright notice required for submissions
\acmConference[Anonymous Submission to ACM CCS 2019]{ACM Conference on Computer and Communications Security}{Due 5 February 2019}{London, UK}
%\acmYear{2019}

%%%%%%%%%%%%%%%%%%%%%%

\newif\ifdraft
%\drafttrue

\usepackage{xcolor}
\usepackage{graphicx,pifont}
\usepackage{adjustbox}
\usepackage{caption}
\usepackage{subcaption}
\usepackage{mathtools}
\usepackage{mathrsfs}
\usepackage{xspace}
\usepackage{url}
\usepackage{tikz}
\usepackage{pifont}
\usepackage{listings}
\usepackage{multirow}
\usepackage{wasysym}
\usetikzlibrary{positioning, trees, arrows}

%\usepackage[font={small}]{caption}
%\usepackage{enumitem}
%\usepackage{cite}
%\usepackage{flushend}
%\usepackage{hyperref}
%\usepackage{caption}



%% Compacting 
%%
%\usepackage[
%all=normal,floats=tight
%,paragraphs=tight
%,wordspacing=tight
%,mathspacing=tight
%,mathdisplays=tight
%]{savetrees}
%\usepackage[compact]{titlesec}


\lstset{
  basicstyle=\ttfamily,
  %numbers=left,
  %numberstyle=\tiny,
  columns=fullflexible,
  showstringspaces=false,
  commentstyle=\color{gray}\upshape
}

\lstdefinelanguage{HTML}
{
  morestring=[b]",
  morestring=[s]{>}{<},
  morecomment=[s]{<?}{?>},
  stringstyle=\color{black},
  identifierstyle=\color{darkblue},
  keywordstyle=\color{blue},
  morekeywords={RegEx,Tag,Type,Input}
}

\colorlet{punct}{red!60!black}
\definecolor{background}{HTML}{EEEEEE}
\definecolor{delim}{RGB}{20,105,176}
\colorlet{numb}{magenta!60!black}

\lstdefinelanguage{json}{
    basicstyle=\small\ttfamily,
    numbers=left,
    numberstyle=\scriptsize,
    stepnumber=1,
    numbersep=8pt,
    showstringspaces=false,
    breaklines=true,
    frame=lines,
    backgroundcolor=\color{background},
    literate=
     *{0}{{{\color{numb}0}}}{1}
      {1}{{{\color{numb}1}}}{1}
      {2}{{{\color{numb}2}}}{1}
      {3}{{{\color{numb}3}}}{1}
      {4}{{{\color{numb}4}}}{1}
      {5}{{{\color{numb}5}}}{1}
      {6}{{{\color{numb}6}}}{1}
      {7}{{{\color{numb}7}}}{1}
      {8}{{{\color{numb}8}}}{1}
      {9}{{{\color{numb}9}}}{1}
      {:}{{{\color{punct}{:}}}}{1}
      {,}{{{\color{punct}{,}}}}{1}
      {\{}{{{\color{delim}{\{}}}}{1}
      {\}}{{{\color{delim}{\}}}}}{1}
      {[}{{{\color{delim}{[}}}}{1}
      {]}{{{\color{delim}{]}}}}{1},
}

\renewcommand\lstlistingname{Specification}

\let\oldding\ding% Store old \ding in \oldding
\renewcommand{\ding}[2][1]{\scalebox{#1}{\oldding{#2}}}% Scale \oldding via optional argument

\newcommand{\zero}{\ding[1.2]{171}\xspace}
\newcommand{\one}{\ding[1.2]{172}\xspace}
\newcommand{\two}{\ding[1.2]{173}\xspace}
\newcommand{\three}{\ding[1.2]{174}\xspace}
\newcommand{\four}{\ding[1.2]{175}\xspace}
\newcommand{\five}{\ding[1.2]{176}\xspace}
\newcommand{\six}{\ding[1.2]{177}\xspace}
\newcommand{\seven}{\ding[1.2]{178}\xspace}
\newcommand{\eight}{\ding[1.2]{179}\xspace}
\newcommand{\nine}{\ding[1.2]{180}\xspace}
\newcommand{\ten}{\ding[1.2]{181}\xspace}

\usepackage{color}
\definecolor{gray}{rgb}{0.4,0.4,0.4}
\definecolor{darkblue}{rgb}{0.0,0.0,0.6}
\definecolor{cyan}{rgb}{0.0,0.6,0.6}

\newcommand{\yes}{\CIRCLE} 
\newcommand{\no}{\Circle} 
\newcommand{\yesNope}{\LEFTcircle} 


\newcommand{\red}[1]{\textcolor{red}{#1}} 

\newcommand{\ad}[1]{\textcolor{red}{Aritra: #1}}
\newcommand{\srdjan}[1]{\textcolor{brown}{Srdjan: #1}}
\newcommand{\todo}[1]{\textcolor{red}{TODO: #1}}
\newcommand{\tocite}{\textcolor{blue}{[cite]}}

%\widowpenalty 200000
%\clubpenalty 200000
%\usepackage[compact]{titlesec}

\newcommand{\name}{\textsc{GuardIOn}\xspace}
\newcommand{\tool}{\name}
\newcommand{\device}{\textsc{Bridge}\xspace}
\newcommand{\server}{\textsc{Server}\xspace}

\newcommand{\toolname}{\name\xspace}


\newcommand{\usb}{USB\xspace}
\newcommand{\bluetooth}{Bluetooth\xspace}
\newcommand{\webusb}{WebUSB\xspace}
\newcommand{\html}{HTML\xspace}
\newcommand{\webbt}{WebBluetooth\xspace}
\newcommand{\http}{HTTP\xspace}
\newcommand{\https}{HTTPS\xspace}
\newcommand{\tls}{TLS\xspace}
\newcommand{\ssl}{\texttt{SSl}\xspace}
\newcommand{\onSelect}{\texttt{onSelect()}\xspace}

%\newcommand{\myparagraph}[1]{{\scshape \bfseries #1.}}
\newcommand{\myparagraph}[1]{\textbf{#1.}}

\newcommand{\webrtc}{\texttt{WebRTC}\xspace}
\newcommand{\js}{JavaScript\xspace}
\newcommand{\relay}{$\mathcal{S}_{relay}$\xspace}
\newcommand{\messenger}{$\mathcal{S}_{messenger}$\xspace}
\newcommand{\serial}{\texttt{serial}\xspace}
\newcommand{\String}{\texttt{string}\xspace}
\newcommand{\integer}{\texttt{integer}\xspace}
\newcommand{\float}{\texttt{float}\xspace}
\newcommand{\menu}{\texttt{menu}\xspace}
\newcommand{\radio}{\texttt{radio button}\xspace}
\newcommand{\Boolean}{\texttt{boolean}\xspace}
\newcommand{\Date}{\texttt{date}\xspace}
\newcommand{\Menu}{\texttt{menu}\xspace}
\newcommand{\Time}{\texttt{time}\xspace}
\newcommand{\mytab}{~~~}
\newcommand{\java}{\textsc{Java}\xspace}

\definecolor{Gray}{gray}{0.85}
\definecolor{LightCyan}{rgb}{0.88,1,1}

\newcommand\MyLBrace[2]{%
  \left.\rule{0pt}{#1}\right\}\text{#2}}

%\newcounter{myExampleCounter}
%\setcounter{myExampleCounter}{-1} % Start with -1.
%\refstepcounter{myExampleCounter}

\newcommand{\Upon}[1]{\textbf{Upon} #1 \newline}

%\newcounter{para}
%\newcommand\mypara{\par\refstepcounter{para}\thepara.\space}
%\newcommand{\myparapara}[1]{\mypara \textbf {\textsc{#1.}}\xspace}

%\newcommand{\redCircle}{$\otimes$}
%\newcommand{\greenCircle}[2][black,fill=white]{\tikz[baseline=-0.5ex]\draw[#1,radius=3pt]
%(0,0) circle ;}
%\newcommand{\yellowCircle}[2][black,fill=black]{\tikz[baseline=-0.5ex]\draw[#1,radius=3pt]
%(0,0) circle ;}

\DeclareGraphicsExtensions{.pdf,.jpeg,.png,.jpg}
%\hyphenation{Integri-Key}

% \newcommand{\redCircle}[][red,fill=red]{\tikz[baseline=-0.5ex]\draw[#1,radius=3pt]
% (0,0) circle ;}
% \newcommand{\greenCircle}[2][green,fill=green]{\tikz[baseline=-0.5ex]\draw[#1,radius=3pt]
% (0,0) circle ;}
% \newcommand{\yellowCircle}[2][red,fill=yellow]{\tikz[baseline=-0.5ex]\draw[#1,radius=3pt]
% (0,0) circle ;}

%------------------------------------------------------------------------------
%                                Space savers.
%------------------------------------------------------------------------------

% This mylist environment indents items, and saves less space than the above.
\newcounter{myctr}
\newenvironment{mylist}{\begin{list}{(\textbf{\arabic{myctr}})}
{\usecounter{myctr}
\setlength{\topsep}{1mm}\setlength{\itemsep}{0.5mm}
\setlength{\parsep}{0.5mm}
\setlength{\itemindent}{0mm}\setlength{\partopsep}{0mm}
\setlength{\labelwidth}{-2mm}
\setlength{\leftmargin}{0mm}}}{\end{list}}

% Space saving List environment for itemizing.
\newenvironment{mybullet}{\begin{list}{$\bullet$}
{\setlength{\topsep}{1mm}\setlength{\itemsep}{0.5mm}
\setlength{\parsep}{0.5mm}
\setlength{\itemindent}{0mm}\setlength{\partopsep}{0mm}
\setlength{\labelwidth}{-2mm}
\setlength{\leftmargin}{0mm}}}{\end{list}}
